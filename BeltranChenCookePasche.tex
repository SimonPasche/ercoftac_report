% 6-8 pages.
% Due May 31, 2016.

\documentclass[11pt]{article}

\usepackage[T1]{fontenc}
\usepackage{lmodern}
\usepackage{hyperref}
% \usepackage{bm}
\usepackage{epsf,graphicx,amssymb,amsmath,amsbsy,authblk}
\usepackage[english]{babel}
\usepackage{psfrag}
\usepackage{fancyhdr}
\usepackage{mathrsfs}
\usepackage{natbib}
\usepackage{xcolor}

% Algebraic rings.
\newcommand*{\Complex}{\mathbb{C}}
\newcommand*{\Ints}{\mathbb{Z}}
\newcommand*{\Reals}{\mathbb{R}}

% Operators.
\renewcommand*{\L}{\mathscr{L}}
\newcommand*{\Lnp}{\L_\mathrm{np}}
\newcommand*{\Lp}{\L_\mathrm{p}}
\newcommand*{\half}{^{1/2}}
\newcommand*{\invhalf}{^{-1/2}}
\newcommand*{\ip}[2]{\left<#1, #2\right>}
\newcommand*{\p}[3][1]{%
    \if#11
    \frac{\partial #2}{\partial #3}
    \else
    \frac{\partial^#1 #2}{\partial #3^#1}
    \fi%
}

% Matrices.
\renewcommand*{\H}{\mathbf{H}}
\newcommand*{\I}{\mathbf{I}}
\newcommand*{\X}{\mathbf{X}}
\newcommand*{\U}{\mathbf{U}}
\newcommand*{\W}{\mathbf{W}}
\newcommand*{\SIGMA}{\mathbf{\Sigma}}

% Vectors.
\renewcommand*{\u}{\mathbf{u}}
\newcommand*{\w}{\mathbf{w}}
\newcommand*{\x}{\mathbf{x}}
\newcommand*{\z}{\mathbf{z}}

% hyperref commands.
\renewcommand*{\sectionautorefname}{Section}

% Other commands.
\newcommand{\KSE}{Kuramoto--Sivashinsky equation}
\newcommand{\kkc}[1]{\textcolor{red}{[KKC: #1]}}

\marginparwidth 0pt
\oddsidemargin 0pt
\evensidemargin 0pt
\marginparsep 0pt
\topmargin -50pt
\textwidth 17cm
\textheight 22.0cm
\headheight 13.6pt
\pagestyle{fancy}

\lhead{{\it Montestigliano Workshop}}
\rhead{{\it 2016 Reports}}

% Bibliography styles.
\bibliographystyle{plainnat}
\setcitestyle{}

\graphicspath{{figures/}}

\begin{document}

\begin{center}
    {\bf \Large Modeling the Kuramoto--Sivashinsky Equation}\\
    \vspace{0.3cm}
    {%
        \large{%
            Victor Beltran,\footnote{Universidad Polit\'ecnica de Madrid}
            Kevin K. Chen,\footnote{University of Southern California}
            Emma Cooke,\footnote{Imperial College London} and
            Simon Pasche\footnote{\'Ecole Polytechnique F\'ed\'erale de Lausanne}%
        }%
    }
\end{center}
\vspace{0.3cm}

\begin{abstract}
    We investigate the nonlinear dynamics and modeling of the \KSE.
    Both the common parallel equation with periodic boundary conditions, as well as a non-parallel equation on an infinite domain, are considered.
    In both equations, we report three cases where the stable solution is the trivial solution, a simple periodic orbit of a single frequency, and a more complex periodic orbit exhibiting multiple harmonics.
    In particular, the parallel periodic equation has analytical eigenvalues and eigenmodes, which allow an simple characterization of the Hopf bifurcation.
    The various types of solutions are demonstrated with phase portraits.
    Next, we compute POD-DEIM modes and reduced-order models for the parallel equation.
    Not only does POD-DEIM capture the limit cycle with a comparatively small number of modes, but also, transient modes correctly capture the limit cycle, and limit cycle modes correctly capture the transients.
\end{abstract}

\section{Introduction}

The \KSE\ is a fourth-order partial differential equation in time and one-dimensional space that arises in many physical phenomena.
The equation was independently discovered by \cite{KuramotoPTP76} in the context of wave propagation in reaction--diffusion systems, and \cite{SivashinskyAA77} in the context of laminar flames.
Since then, the equation has also been derived for thin film flows, interfacial instabilities, drift waves in plasmas \citep[see][]{KevrekidisSIAMJAM90}, Tollmien--Schlichting waves in flat-plate boundary layers \citep{FabbianeAMR14}, and many more applications.

One of the features that makes the \KSE\ a common model problem is the complex hierarchy of attractors for different equation parameters.
The range of stable solutions, as discussed by \cite{KevrekidisSIAMJAM90}, includes fixed points, simple periodic orbits, more complex periodic orbits (e.g., exhibiting multiple frequencies), quasi-periodic orbits, and ultimately chaos.
Therefore, many attempts have been made to characterize the equation's complex dynamics using simple reduced-order models.
In particular, some works \citep[e.g.,][]{AubrySIAMJSC93, RowleyPD00} have successfully taken advantage of symmetry properties to improve proper orthogonal decomposition (POD) models.

Proper orthogonal decomposition---also known as principal component analysis and Karhunen--Lo\`eve expansion---is a very common data-based modal decomposition, and is discussed in works by \cite{SirovichQAM87} and \cite{HolmesTCSDSS}.
Given a set of data snapshots, the technique extracts orthogonal modes that optimally capture the energy of the input data.
These modes are then typically used in Galerkin models to form reduced order models.
Despite the popularity of POD-based reduced order models, the technique suffers from a number of shortcomings.
For instance, the kinetic energy of POD modes do not necessarily correlate to their importance in the dynamics, especially if some low-energy mode may trigger instabilities.
The choice of modes to retain in reduced-order models, therefore, may be difficult to determine \citep[see, e.g.,][]{IlakPF08}.
In addition, POD models generally have no stability guarantees.
Finally, the construction of nonlinear terms in POD models may be computationally expensive.

The discrete empirical interpolation method \citep[DEIM;][]{ChaturantabutIEEECDC09, ChaturantabutRice09a} addresses the last concern by intelligently selecting discrete points in the domain for which the nonlinear term is computed, and neglecting all other points.
The DEIM method has been shown to be very computationally efficient and to create very accurate POD models.
Although error bounds have been derived for DEIM, the precise reasons for its success have not been rigorously proven.

In this report, we cover two topics related to the \KSE.
First, we use linear stability theory and computation to characterize the equation's simple attractors.
Second, we demonstrate the success of POD-DEIM reduced-order models for this equation.
The \KSE\ and POD-DEIM are first reviewed in \autoref{sec:theory}.
Next, the attractors are described in \autoref{sec:attractors}, and the application of POD-DEIM is demonstrated in \autoref{sec:pod-deim}.
Finally, we summarize our work in \autoref{sec:conclusions}.

\section{Theory}
\label{sec:theory}

\subsection{The \KSE}

In this report, we cover two different forms of the \KSE.
In the parallel and periodic equation, we consider the streamwise displacement $x$ on the periodic domain $[-1, 1]$ and time $t$.
We also define positive real constants $U, \mu_0, \gamma$ respectively corresponding to the advection velocity, anti-diffusion, and short-wavelength diffusion.
The dependent variable $w(x, t) : \Reals \times \Reals \to \Reals$ obeys the nonlinear equation
\begin{equation}
    \label{eq:ks-parallel}
    \p{w}{t} = \Lp w - w \p{w}{x},
\end{equation}
where the linear operator $\Lp$ is given by
\begin{equation}
    \label{eq:ks-parallel-linear}
    \Lp := - \left(U \p{}{x} + \mu_0 \p[2]{}{x} + \gamma \p[4]{}{x}\right).
\end{equation}

The role of each term can be seen by examining a snapshot in time of the form $w = e^{i \pi n x}$ for wavenumbers $n \in \Ints$.
The two first derivative terms correspond to advection at the speed $U + w$, as in the Burger's and Navier--Stokes equation;
the second derivative term evaluates to $(\pi n)^2 \mu_0 w$ and is therefore antidiffusive and destabilizing;
and the fourth derivative term evaluates to $- (\pi n)^4 \gamma w$ and is therefore diffusive, especially at short wavelengths.

The non-parallel form of the \KSE\ that we will consider is
\begin{equation}
    \label{eq:ks-nonparallel}
    \p{w}{t} = \Lnp w - w \p{w}{x},
\end{equation}
where given $\mu_0(x) = e^{-x^2}$,
\begin{equation}
    \label{eq:ks-nonparallel-linear}
    \Lnp := - \left(
        U \p{}{x} + \p{}{x} \left(\mu_0(x) \p{}{x}\right) + \gamma \p[4]{}{x}
    \right).
\end{equation}
For this form, we consider $x \in \Reals$ and assume that $w$ is well-behaved at $x \to \pm \infty$.
We note that in this form, the anti-diffusive second derivative term only acts in a narrow domain near $x = 0$.

For both forms, we remark that the $U \partial / \partial x$ term is actually optional---$w$ could be replaced with $U + w$ and the $U \partial / \partial x$ term removed, and the equations will remain the same.
In fact, most works omit the $U \partial / \partial x$ term in their analyses.
The purpose of this term is to allow the base flow to be fixed at $w(x, t) = 0$, which corresponds to an advection speed of any arbitrary $U$.

\subsection{POD-DEIM}

\kkc{%
    I'm consciously being a bit verbose here.
    It's not hard to cut this section short to make more room for results, if need be.%
}

Proper orthogonal decomposition \citep{SirovichQAM87, HolmesTCSDSS} is a method of identifying dominant directions in a size $m$ set of data $\{\x_k\}_{k=1}^m$, with $\x_k \in \Complex^n$ for $k = 1, \ldots, m$.
(It is common in many problems to restrict $\x_k$ to $\Reals^n$.)
The modes are ranked in a particular order so the first $r$ modes maximize the orthogonal projection of the data onto the span of those modes for each $r = 1, \ldots, m$.
Given an inner product $\ip{\cdot}{\cdot} : \Complex^n \times \Complex^n \to \Complex$, we therefore write the POD modes $\{\u_j\}_{j=1}^m$ as the set of vectors $\u_j \in \Complex^n$ for $j = 1, \ldots, m$ such that
\begin{subequations}
    \begin{align}
        \sum_{k=1}^m \left\|\x_k - \sum_{j=1}^r \ip{\x_k}{\u_j} \u_j \right\|_2^2 \quad & \text{is minimized for each } r = 1, \ldots, m, \\
        \ip{\u_j}{\u_k} = 0, \quad & j \ne k.
    \end{align}
\end{subequations}
Note that this expression implies an ordering of the POD modes: the first mode is chosen to maximize the orthogonal projection of the data onto it; the second mode is then chosen to maximize the projection onto the first two modes; and in general, the $r$th mode is then chosen to maximize the projection onto the first $r$ modes for increasing $r$.
In reduced-order modeling, it is common to truncate the POD mode set and only consider the proper subset $\{\u_j\}_{j=1}^r$ for $1 \le r \ll m$, with the assumption that most interesting features of $\{\x_k\}_{k=1}^m$ are contained in only $r$ modes.

The computation of POD is very straightforward.
Let us assume the common case where $m \le n$ and $\ip{\z_j}{\z_k} := \z_k^* \z_j$ for $\z_j, \z_k \in \Complex^n$, and $(\cdot)^*$ indicates the conjugate transpose.
The economy-sized singular value decomposition (SVD) of the data matrix $\X := \begin{bmatrix} \x_1 & \ldots & \x_m \end{bmatrix} \in \Complex^{n \times m}$ is
\begin{equation}
    \label{eq:svd}
    \X = \U \SIGMA \W^*,
\end{equation}
where $\U \in \Complex^{n \times m}$ and $\U^* \U = \I$, $\SIGMA \in \Reals^{m \times m}$ is a diagonal matrix with $\{\sigma_k\}_{k=1}^m$ on the diagonal such that $\sigma_1 \ge \cdots \ge \sigma_m \ge 0$, and $\W \in \Complex^{m \times m}$ and $\W^* \W = \W \W^* = \I$.
The POD modes are simply the columns of $\U$; that is, $\U = \begin{bmatrix} \u_1 & \cdots & \u_m \end{bmatrix}$.

When representing data on non-uniform grids, it is common to use the weighted inner product $\ip{\z_j}{\z_k} := \z_k^* \H \z_j$, where typically, $\H \in \Reals^{n \times n}$ is a diagonal matrix with the positive grid sizes on the diagonal.
In this case, the orthogonality of the POD modes with respect to this inner product is equivalent to $\U^* \H \U = \I$ instead of $\U^* \U = \I$.
A quick examination of~\eqref{eq:svd} reveals that one can write
\begin{equation}
    \label{eq:svd-scaled}
    \H\half \X = \H\half \U \SIGMA \W^*,
\end{equation}
where $\H\half \U$ is unitary.
Therefore, the POD algorithm is modified so that one computes the SVD of $\H\half \X$ and left-multiplies the left singular matrix by $\H\invhalf$ to recover the matrix of POD modes.

We also remark that in the typical case that $m \ll n$, it is more computationally efficient to perform the $m \times m$ eigendecomposition
\begin{equation}
    \X^* \H \X = \W \SIGMA^2 \W^*
\end{equation}
than the $n \times m$ SVD~\eqref{eq:svd-scaled}.
The POD modes can then be recovered from~\eqref{eq:svd-scaled} by $\U = \X \W \SIGMA^{-1}$.
This technique is known as the method of snapshots.

For linear systems of the form 

\section{Numerical methods}
\label{sec:methods}

\cite{WeidemanACMTMS00}
\cite{HouJCP07}
\cite{HechtJNM12}

\section{Characterizations of attractors}
\label{sec:attractors}

\subsection{Parallel equation}

% Include discussion of Hopf bifurcation

\subsection{Non-parallel equation}

\section{POD-DEIM models of the parallel equation}
\label{sec:pod-deim}

\section{Conclusions}
\label{sec:conclusions}

\section{Acknowledgements}

We would like to thank ERCOFTAC for funding and providing the 2016 Montestigliano Spring School.
We are also deeply grateful for Denis Sipp, Peter Schmid, and Shervin Bagheri for leading the school.
K. K. C. was supported by the Viterbi Postdoctoral Fellowship through the Viterbi School of Engineering at the University of Southern California.
\kkc{Victor, Emma, Simon---add your funding sources and other acknowledgements here.}

\bibliography{BeltranChenCookePasche}

\end{document}

%%% Local Variables:
%%% mode: latex
%%% TeX-master: t
%%% End:
