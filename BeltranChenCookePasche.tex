% 6-8 pages.
% Due May 31, 2016.

\documentclass[11pt]{article}

\usepackage[T1]{fontenc}
\usepackage{lmodern}
\usepackage{epsf,graphicx,amssymb,amsmath,amsbsy,authblk}
\usepackage[english]{babel}
\usepackage{psfrag}
\usepackage{fancyhdr}
\usepackage{natbib}
\usepackage{xcolor}

% Operators.
\newcommand*{\p}[3][1]{%
    \if#11
    \frac{\partial #2}{\partial #3}
    \else
    \frac{\partial^#1 #2}{\partial #3^#1}
    \fi%
}

% Other commands.
\newcommand{\kkc}[1]{\textcolor{red}{[KKC: #1]}}

\marginparwidth 0pt
\oddsidemargin 0pt
\evensidemargin 0pt
\marginparsep 0pt
\topmargin -50pt
\textwidth 17cm
\textheight 22.0cm
\headheight 13.6pt
\pagestyle{fancy}

\lhead{{\it Montestigliano Workshop}}
\rhead{{\it 2016 Reports}}

% Bibliography styles.
\bibliographystyle{plainnat}
\setcitestyle{}

\graphicspath{{figures/}}

\begin{document}

\begin{center}
    {\bf \Large Modeling the Kuramoto--Sivashinsky Equation}\\
    \vspace{0.3cm}
    {%
        \large{%
            Victor Beltran,\footnote{Universidad Polit\'ecnica de Madrid}
            Kevin K. Chen,\footnote{University of Southern California}
            Emma Cooke,\footnote{Imperial College London}
            Simon Pasche\footnote{\'Ecole Polytechnique F\'ed\'erale de Lausanne}%
        }%
    }
\end{center}
\vspace{0.3cm}

\begin{abstract}
    We investigate the nonlinear dynamics and modeling of the Kuramoto--Sivashinsky equation.
    Both the common parallel equation with periodic boundary conditions, as well as a non-parallel equation on an infinite domain, are considered.
    In both equations, we report three cases where the stable solution is the trivial solution, a simple periodic orbit of a single frequency, and a more complex periodic orbit exhibiting multiple harmonics.
    In particular, the parallel periodic equation has analytical eigenvalues and eigenmodes, which allow an simple characterization of the Hopf bifurcation.
    The various types of solutions are demonstrated with phase portraits.
    Next, we compute POD-DEIM modes and reduced-order models for the parallel equation.
    Not only does POD-DEIM capture the limit cycle with a comparatively small number of modes, but also, transient modes correctly capture the limit cycle, and limit cycle modes correctly capture the transients.
\end{abstract}

\section{Introduction}

\cite{KuramotoPTP76}
\cite{SivashinskyAA77}
\cite{ChaturantabutRice09a}
\cite{ChaturantabutIEEECDC09}
\cite{HolmesTCSDSS}
\cite{SirovichQAM87}
\cite{KevrekidisSIAMJAM90}
\cite{AubrySIAMJSC93}
\cite{RowleyPD00}

\section{Theory}

\subsection{The Kuramoto--Sivashinsky Equation}

\subsection{POD-DEIM}

\section{Numerical methods}

\cite{WeidemanACMTMS00}
\cite{HechtJNM12}

\section{Characterizations of attractors}

\subsection{Parallel equation}

% Include discussion of Hopf bifurcation

\subsection{Non-parallel equation}

\section{POD-DEIM models of the parallel equation}

\section{Conclusions}

\section{Acknowledgements}

We would like to thank ERCOFTAC for funding and providing the 2016 Montestigliano Spring School.
We are also deeply grateful for Denis Sipp, Peter Schmid, and Shervin Bagheri for leading the school.
K. K. C. was supported by the Viterbi Postdoctoral Fellowship through the Viterbi School of Engineering at the University of Southern California.
\kkc{Victor, Emma, Simon---add your funding sources and other acknowledgements here.}

\bibliography{BeltranChenCookePasche}

\end{document}

%%% Local Variables:
%%% mode: latex
%%% TeX-master: t
%%% End:
